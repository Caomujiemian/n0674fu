% 毒舌にもかまわずグイグイいく
\subsection{无惧毒舌不断攻略}   % 这是标题,设计自己的格式

%  次の日の昼休み。
到了第二天下午。

%  学校の廊下を歩いていると、予期せぬ人物が直哉の前に立ちはだかった。
走在学校的走廊里,一名意外的人突然挡在直哉面前。\\

% 「あなたが『笹原直哉くん』よね。昨日はどうも」
「你就是『笹原直哉君对吧』。昨天的事谢谢你了」\\

%  腰まで届く髪は銀。宝石のようにきらめく瞳は海の色。
%  その面立ちは整いすぎていて、よくできたCGと言われても納得するほどだ。肌は透けるように白く、小さな唇からこぼれ落ちる声も鈴を転がすように澄んでいる。\\
她有一头及腰的银色长发,以及宝石般闪烁的海色眼眸;容貌极为端正,说是精美的CG也不为过;肌肤雪白通透,小小的嘴唇中发出的声音也像银铃一样清亮。\\

%  ただし、こちらに向ける視線はやけに鋭い。
只是,她看过来的眼神却无比锐利。

%  小柄なその体からは殺気がほとばしっており、腕を組んで仁王立ちするその姿は、まさに金剛力士像だ。\\
娇小的身子里释放出杀气,再加上那抱着胳膊仁王般的站姿,简直就是个金刚力士像。\\

%  おかげで廊下に溢れるほかの生徒たちはざわざわする。
结果就是,挤满走廊的其他学生喧闹起来。

%  一緒にいたチャラめの男子生徒、|河野巽《こうのたつみ》も目を丸くして直哉に耳打ちした。
在一起的轻浮男生河野巽也睁大眼睛,向直哉耳语。\\

% 「おいおい直哉、おまえ……『猛毒の白雪姫』と何かあったのかよ?」
「喂,我说直哉你啊……是和『剧毒的白雪公主』发生了什么吗?」 % 称号待定

% 「ああ、うん。昨日ちょっとな」
「嗯,昨天确实发生了点事情」\\

%  直哉は鷹揚にうなずいてみせる。
直哉大大方方地点头。

%  あのときは顔を見なかったが……長い銀髪で、おそらく彼女だろうとは察しがついていた。
当时他没看到对方的脸……不过,从这头银色长发来看,直哉推测出就是她没错了。\\

% (ただ、こうやって再エンカウントするとは思わなかったよなあ)
(可是谁想得到,还能这样再次碰到她啊)\\

%  彼女の名前は|白金小雪《しろがねこゆき》。
%  直哉と同じく、大月学園の二年生だ。
%  容姿端麗かつ頭脳明晰、おまけにスポーツ万能というその評判は、別のクラスである直哉の耳にも届いていた。
她叫白金小雪,和直哉一样,在大月学园读二年级。听说她容貌端正、头脑清晰,而且还精通各种体育运动,这些评价甚至传到了不在同班的直哉的耳朵里。

%  ただし、どちらかというと広まっているのは賞賛だけではなく、悪名の方が多かった。
不过,传开的其实不只有赞赏,反倒是恶名更多一些。

%  ぼんやりする直哉を前にして、小雪は淡々と告げる。\\
面对傻站着的直哉,小学淡淡地说:\\

% 「昨日はどうもありがとう。お礼を言いそびれちゃったからわざわざ探したのよ。制服だったから、同じ学校だってことは分かったし」
「昨天多谢了,当时没机会道谢,这不来找你了嘛。你那时也穿着校服,所以看得出是同一所学校的」

% 「別にお礼なんていいのに」
「不用那么客气啦」

% 「そういうわけにはいかないわ」
「那怎么行」\\

%  小雪はじろりと直哉をねめつけて、長い髪をかきあげてみせる。
小雪直勾勾地瞪着直哉,然后扬起那头长发。\\

% 「あの程度のことで、恩を売ったなんて思われちゃ困るもの。そうでなきゃ、この私がただの男子生徒Aにわざわざ声なんかかけるわけないでしょ」
「别以为那样就算卖我一个人情了。要不是怕你这样,我怎么会找一个普通男生A搭话呢」

% 「はあ」
「唉」\\

%  完全無欠の美少女、白金小雪の唯一の欠点。
%  それがこの毒舌だ。
这是完美无缺的美少女白金小雪唯一的缺点:毒舌。

%  入学から一年あまりが経過して、これまで何人もの男子生徒がアタックし、その全員が彼女の苛烈極まりない口撃によってノックアウトさせられたという。
%  結果、ついたあだ名が『猛毒の白雪姫』。
入学一年有余,传说有数名男生向她发起进攻,结果全部被她喷得体无完肤。她进而得名『剧毒的白雪公主』。\\

% 「今日もキッツイなあ、猛毒の白雪姫……」
「今天也一样狠啊,剧毒的白雪公主……」

% 「ねえ……何があったか知らないけど、言い方ってものがあるよね」
「那个……我是不知道发生了什么啦,不过有话要好好说嘛」\\

%  ギャラリーたちも眉をひそめ、ひそひそと言葉を交わし合う。
%  だがしかし、小雪は目つきをさらに鋭くして続ける。
围观群众皱起眉头窃窃私语,小雪的眼神却愈发锐利:\\

% 「昨日はたしかに少し怖かったけど……あなたが手を出さなくても、私ひとりでどうにかなったんだから。くれぐれも、白馬の王子さま気取りはやめてちょうだいね。私、借りを作るのは好きじゃないの」
「昨天我确实有那么点害怕……但就算没有你帮忙,我一个人也能想办法解决。你可千万别把自己当白马王子了,我这人不喜欢欠别人什么的」

% 「おお、わかった」
「哦,我明白了」\\

%  直哉はあっさりとうなずく。
%  彼女の言いたいことは、よーーーく理解した。
直哉爽快地点头,他完——全明白了她的意思。\\

% 「つまり白金さんは俺にお礼がしたいから、放課後どこかに誘いたいんだな?」
「总之,白金你是想给我谢礼,放学后要邀请我去别的地方玩玩?」

% 「…………は?」
「…………啥?」

% 『…………はあ?』
『…………哈?』\\

%  小雪だけでなく、周囲の生徒たちも目を丸くした。
%  おおむね『何を言ってるんだこいつは』という反応だ。
不只是小雪,周围的学生也都瞪大了眼睛。他们的表现大致上就是:『这货说啥呢』。

%  だがしかし、すぐに小雪の様子がおかしくなる。一瞬で耳の先まで真っ赤に染まり、裏返った声を上げた。
不过,小雪很快就变得不正常了,她一瞬间脸红到了耳根,发出尖叫:\\

% 「い……いったい何を言ってるのよ!? 今の私のセリフを聞いて、どうしてそんな結論になるわけ!?」
「你……你这是说的什么话!?你有听我说吗!?为什么得到那种结论!?」

% 「いやだって、わかりやすいから」
「那还不是你太好懂了嘛」\\

%  直哉はあっさり告げるしかない。
直哉只得直截了当地挑明。\\

% 「『怖かった』ってのは本当だろ。あとはほとんど強がりだ」
「『害怕』是真的吧,后面基本都在逞强」

% 「っ……!」
「……!」

% 「で、『借りを作りたくない』ってのも本当だけど、ちょっとニュアンスが違う。本音は『お礼がしたい』だ」\\
「然后『不想欠别人什么』也是真的,不过话里有话,其实你是『想回礼』」\\

% そして、昼休みもそろそろ終わる。
% 小雪が本当にお礼がしたいと思っているのなら、放課後しか時間はないだろう。
考虑到午休快要结束,如果小雪是真的想要回礼,那就只有放学后才可以了。

% これくらい、小雪の口ぶりや態度、状況などを鑑みれば誰にでもわかることだった。
从小雪的语气、态度和状况来看,是个人都能推理出这些。

% ぽかんと言葉を失う小雪に、直哉はつらつらと畳み掛ける。\\
向着呆呆说不出话的小雪,直哉一个劲儿说了下去:\\

% 「今日はちょうどバイトがないんだ。部活もやってないし、放課後は空いてる。白金さん、どうする?」
「今天我正好不做兼职,也不参加社团,放学后有时间。白金,怎么样?」

% 「だ、だから、私は……うっ、ううう……!」\\
「所、所以,我……唔、唔唔唔……!」\\

% 小雪はぷるぷると震え、俯いてしまう。
小雪打着哆嗦,低下了头。

% そのまましばらく待ってみれば……彼女はぼそぼそと小声でこぼす。\\
又等了一会儿……她小声嘀咕起来:\\

% 「あの、もしよかったら…………で、待ってるから……だから、その……」
「那个,你方便的话,…………等你……所以,那个……」

% 「うん。わかった、放課後に正門前で待ち合わせだな。了解」
「嗯,知道了。放学后在正门前碰头对吧,好的」

% 「なんでちゃんと聞こえるのよ!? そこは普通、聞こえなくて聞き返すってのがセオリーじゃなくって!?」
「你怎么给听清楚了啊!?这种时候一般不是该听不清再问一遍的吗!?」

% 「いや俺、生まれてこのかた聴力検査で引っかかったことないからさ」
「不是,我从出生到现在,从来没被听力检查难住来着」

% 「ぐううっ……! こ、この……!」
「唔……!你、你这……!」

% 「この?」
「嗯?」

% 「笹原くんの……健康優良児ぃぃぃぃ!」\\
「笹原君你个……三好学生——!」\\

% 褒めているとしか思えない捨て台詞を残し、小雪は真っ赤な顔のまま逃げ出してしまった。\\
小雪抛下一句怎么想都是夸奖的话,然后满脸通红地逃走了。\\

% 「……白金さんって、案外……」
「……没想到白金……」

% 「ねえ……」
「是啊……」

% 「可愛いところもあるんだねえ……」\\
「也有可爱的地方啊……」\\

% 彼女が消えた方へと、ギャラリーたちは生温かい目を向ける。
围观群众一脸祥和地看向她消失的方向。

% そんななか、友人の河野がぽんっと直哉の肩を叩いた。
这时,直哉的朋友河野拍了拍他的肩膀。

% 呆れたような、笑いを堪えるような顔で言うことには――。\\
他傻眼的表情,好像是在憋着笑:\\

% 「おまえのその読心スキル、今日も絶好調だな」
「你的读心术,今天也是状态绝佳啊」

% 「これくらいみんなわかるだろ?」\\
「这点事,大家都能明白的吧?」\\

% 直哉は不思議そうに首をかしげるだけだった。\\
——而直哉却只是感到疑惑不解。\\

\vspace{2\baselineskip}

% これはやたらと察しのいい少年が、毒舌クーデレ少女に完勝を続ける物語。
这是一个观察力太好的少年,不断完胜毒舌冷娇少女的故事。
